\documentclass{scrartcl}% siehe <http://www.komascript.de>
\usepackage[utf8]{inputenc}
\usepackage[T1]{fontenc}

\usepackage{babel}
\usepackage{blindtext}
\usepackage{microtype}
\usepackage{xcolor}
\usepackage{csquotes}
\usepackage{paralist}
\usepackage{graphicx}

\usepackage{hyperref}
\hypersetup{
bookmarks=true, % show bookmarks bar
unicode=false, % non - Latin characters in Acrobat’s bookmarks
pdftoolbar=true, % show Acrobat’s toolbar
pdfmenubar=true, % show Acrobat’s menu
pdffitwindow=false, % window fit to page when opened
pdfstartview={FitH}, % fits the width of the page to the window
pdftitle={My title}, % title
pdfauthor={Author}, % author
pdfsubject={Subject}, % subject of the document
pdfcreator={Creator}, % creator of the document
pdfproducer={Producer}, % producer of the document
pdfkeywords={keyword1, key2, key3}, % list of keywords
pdfnewwindow=true, % links in new window
colorlinks=true, % false: boxed links; true: colored links
linkcolor=blue, % color of internal links
filecolor=blue, % color of file links
citecolor=blue, % color of file links
urlcolor=blue % color of external links
}

\begin{document}
% ----------------------------------------------------------------------------
% Titel (erst nach \begin{document}, damit babel bereits voll aktiv ist:
\titlehead{}% optional
\subject{Als Testfile}% optional
\title{ein kleiner Adventsgruß}% obligatorisch
\subtitle{im kälter werdenden Winter}% optional
\author{von Ruth an dich und euch}% obligatorisch
\date{ohne Bratäpfel, heißen Tee und Stollen}% sinnvoll
\publishers{sondern mit \LaTeX}% optional
\maketitle% verwendet die zuvor gemachte Angaben zur Gestaltung eines Titels
% ----------------------------------------------------------------------------
% Inhaltsverzeichnis:
\tableofcontents

\listoffigures

\begin{center}
\includegraphics[width=0.3\textwidth]{Bilder/Schneeweihmann01}\captionof{figure}{Ein glücklicher Weihnachtsschneemann}\label{fig:Schneeweihmann01}
\end{center}

\begin{center}
\includegraphics[width=.7\textwidth]{Bilder/Blumbutterfly1}\captionof{figure}{Gelber Schmetterling mit Blumen}\label{fig:Blumbutterfly1}
\end{center}
% ----------------------------------------------------------------------------
% Gliederung und Text:
\section{\LaTeX kann alles}
Auch den Advent begrüßen. 
\section{Und wir?}
Immer mehr auch \LaTeX. 
\subsection{Motivation}
Hohe Motivation verlängert die Lernzeit.
\subsection{Freude über Erlerntes}
Partyfeiern verkürzt sie. 
\section{Fazit}
Immer ein paar Cookies bereithalten.  

\end{document}